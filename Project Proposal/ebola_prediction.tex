\documentclass[12pt, journal,onecolumn]{IEEEtran}
\usepackage{cite}
\usepackage{graphicx}
\graphicspath{{figure/}}
\usepackage{array}
\usepackage{subfigure}
\usepackage{stfloats}
\usepackage{url}
\usepackage{nicefrac}
\usepackage{amssymb}
\usepackage{amsmath}
%\interdisplaylinepenalty=2500
\usepackage{amsfonts}
\usepackage{amssymb}
\usepackage{relsize}
\usepackage[pdfpagelabels]{hyperref}
\usepackage{float}
\usepackage{color}
%\usepackage{kbordermatrix}


\floatstyle{ruled}
\newfloat{algorithm}{tbp}{loa}
\floatname{algorithm}{Algorithm}

\newtheorem{theorem}{Theorem}
\newtheorem{corollary}{Corollary}
\newtheorem{proposition}{Proposition}
\newtheorem{definition}{Definition}
\newtheorem{lemma}{Lemma}

\newcommand{\norm}[1]{\left\lVert#1\right\rVert}
\newcommand{\abs}[1]{\left\lvert#1\right\rvert}
\newcommand{\inner}[1]{\left\langle#1\right\rangle}
\def\b#1{\mathbf{#1}}
\def\t#1{\text{#1}}


\begin{document}

\title{Predicting 2014 Ebola Outbreak in West Africa using Network Analysis}
\author{Shafi Bashar {\em SUID: 06018208}, Mike Percy {\em  SUID: }, Romit  Singhai {\em  SUID: 06005567}
%\thanks{Paper approved by A. Yener, the Editor for Information Theory and Physical-Layer Security of the IEEE Communications Society. Manuscript received April 5, 2011; revised December 21, 2012 and June 16, 2012.}
%\thanks{This material is based upon work supported by the National Science Foundation under Grants CCF-0830706 and CNS-091751. The work of C. Xiao was supported in part by National Science Foundation under Grants CCF-0915846 and ECCS-1231848.}
%\thanks{S. Bashar and Z. Ding are with the Department of Electrical and Computer Engineering, Univ. of California, Davis, CA, 95616, USA. e-mail: \{shafiab, ding\}@ece.ucdavis.edu.}
%\thanks{C. Xiao
%is with the Department of Electrical and Computer Engineering, Missouri University of Science and Technology,
%Rolla, MO, 65409, USA. e-mail: xiaoc@mst.edu.}
}
\maketitle

%%%%%%%%%%%%%%%%%%%%%%%%%%%%%%%%%%%%%%%%%%%%%%%%%%%%%%%%%
%\begin{abstract}
%Few words about abstract -- some favorite quote etc :)
%
%\em{Mike, Romit : please extend this section}
%\end{abstract}
%%%%%%%%%%%%%%%%%%%%%%%%%%%%%%%%%%%%%%%%%%%%%%%%%%%%%%%%%

%\begin{IEEEkeywords}
%Wiretap channel, eavesdropping, information-theoretic security, secrecy rate, finite-alphabet input.
%\end{IEEEkeywords}

%%%%%%%%%%%%%%%%%%%%%%%%%%%%%%%

\section{Introduction}
\label{sec:introduction}
%{\em{Mike, Romit : please extend this section. this section can also be combined with abstract}}

\bigskip
\IEEEPARstart{C}{urrent} Ebola outbreak in West Africa is the worst in history, and shows no signs
of abating. At the time of writing, the number of cases in Liberia continues to grow exponentially,
while linear growth continues in Sierra Leone.




Computer models are incredibly helpful in curbing an outbreak. They can help agencies like WHO to predict the resources and interventions needed
to stem the outbreak.Most of the outbreak models  are compartmental models and are used  for calculating and reducing the virus effective reproductive rate.



We organize the rest of the paper as follows. In section \ref{sec:ReactionPaper}, we provide a survey of previous relevant work in the area of epidemic prediction.
Finally, in Section \ref{sec:ProjectProposal}, we provide the project proposal and direction of research.

%%%%%%%%%%%%%%%%%%%%%%%%%%%%%%%%
\section{Reaction Paper}
\label{sec:ReactionPaper}
\subsection{SIR model for Epidemiology}
\label{SubSec:SIR}
The basis of majority of research in epidemiological theory is the based on compartmental model. In a compartmental model, to model the progress of an epidemic in a large population, the individuals in the population compartmentalized according to the state of the disease. The most widely used compartmental model is the SIR model introduced in \cite{very_old_paper}. In SIR model, three compartments or disease state for an individual is used:

\begin{itemize}
\item \textbf{S (Susceptable) : } These are individuals before catching the disease. They are susceptible to infection following a contact with infectious individuals.
\item \textbf{I (Infectious) : }These are individuals who have caught the disease. They are infectious and have some probability of infecting each of their susceptible neighbors.
\item \textbf{R (Recovered) : }These are individuals who have experienced the full infectious period. These nodes are removed from consideration, since they no longer pose threat of future infection.
\end{itemize}

The changes among these states over time are represented by a set of differential equations. In order to capture the dynamics of disease spread over time, a population-wide random mixing model is assumed. In random mixing model, the population mixes at random, so that each individual has a small and equal chance of coming in contact with any other individual. A basic reproductive number $R_0$ is defined as the average number of secondary cases generated by a primary case in a pool of mostly susceptible individuals and is an estimate of epidemic growth at the start of an outbreak if everyone is susceptible.


\subsection{Paper \cite{chowell2004basic}}
In  \cite{chowell2004basic}, the authors model the effect of Ebola outbreaks in Congo 1995 and Uganda 2000 using compartmental model similar to the SIR model in Section \ref{SubSec:SIR}. However, a distinct feature of Ebola disease is, individuals exposed to the virus who become infectious do so after a mean incubation period. In order to reflect this feature, in \cite{chowell2004basic}, the basic SIR model is modified by adding an additional compartment ``Exposed". The modified SIR model, i.e. the SEIR model presented in  \cite{chowell2004basic} is reproduced in Figure \ref{Fig:SEIR_model}.

In SEIR model, susceptable(S) individuals in contact with the virus enter the exposed(E) state at a rate of $\beta I / N$. Here,

\begin{eqnarray*}
\beta &=& \text{transmission rate per person per day}\\
N &=& \text{total effective population size}\\
\dfrac{I}{N} &=& \text{probability that a contact is made with a infectious individual, assuming random uniform mixing}
\end{eqnarray*}

The exposed(E) individuals undergo an average incubation period of $1/k$ days before progressing to the infectious(I) state. The exposed state is assumed to be asymptomatic as well as uninfectious. Infectious(I) individual move to R state, i.e. either recovered or death at a rate of $\gamma$.

The following set of differential equations are used to represent this model:

\begin{eqnarray}
\label{Eq:SEIR}
\dfrac{dS}{dt}	&=&	\dfrac{-\beta SI}{N}\nonumber\\
\dfrac{dE}{dt}	&=&	\dfrac{\beta SI}{N}-kE\nonumber\\
\dfrac{dI}{dt}	&=&	kE-\gamma I\nonumber\\
\dfrac{dR}{dt}	&=&	\gamma I\nonumber\\
C	&=&	kE\nonumber\\
 \end{eqnarray}
 
 Here, $S$, $E$, $I$, and $R$ denote the number of susceptible, exposed, infectious and removed individual at time $t$. In the equations, for simplicity of notations, we removed the dependence of $t$. $C$ is not an epidemiological state, however is useful to keep track of the cumulative number of cases from the time of the onset of the outbreak.
 
 In order to model the effect of intervention on the spread of the disease, in the above model, the transmission rate $\beta$ is modeled as a function of time. At the initial phase of the outbreak, before intervention, $\beta$ is parameterized by $\beta_0$. After intervention, the value of $\beta$ transitions from $\beta_0$ to $\beta_1$, $\beta_0>\beta_1$ as follows:
 
 \[
\beta(t)=\begin{cases}
\beta_{0} & t<\tau\\
\beta_{1}+(\beta_{0}-\beta_{1})\exp\left(-q\left(t-\tau\right)\right) & t\ge\tau
\end{cases}
\]

Here, $\tau$ is the time at which the interventions start and $q$ control the rate of transmission from $\beta_0$ to $\beta_1$.

The Ebola data for Congo 1995 and Uganda 2000 outbreak were represented as $(t_i,y_i)$, $i=1,2,\ldots,n$ where $t_i$ represents $i$th reporting time and $y_i$ the cumulative number of infectious cases from the beginning of the outbreak of to time $t_i$.  The model parameters $\Theta=(\beta_0,\beta_1,k,q,\gamma)$ were estimated using least-square fit by fitting these data to the cumulative number of cases $C(t,\Theta)$ in Equation \ref{Eq:SEIR}. The initial condition and appropriate of range of the parameters were taken from Empirical studies, e.g. an incubation period between $1$ and $21$ days and infectious period between $3.5$ and $10.7$ days were assumed. Once the parameters are estimated, the basic reproductive number is calculated using the following formula

\begin{equation}
R_0 = \dfrac{\beta_0}{\gamma}
\end{equation}


In addition of calculating $R_0$, \cite{chowell2004basic} also proposed an analogous continuous time Markov chain model based on the estimated parameters. The transition rates were defined as follows:

\begin{center}
\begin{tabular}{|c|c|c|}
\hline 
Event & Effect & Transition Rate\tabularnewline
\hline 
\hline 
Exposure & (S,E,I,R)$\to$(S-1,E+1,I,R) & $\beta SI/N$\tabularnewline
\hline 
Infection & (S,E,I,R)$\to$(S,E-1,I+1,R) & $kE$\tabularnewline
\hline 
Removal & (S,E,I,R)$\to$(S,E,I-1,R+1) & $\gamma I$\tabularnewline
\hline 
\end{tabular}
\end{center}


The event times $0<T_1<T_2<\ldots$ at which an individual moves from one state to another are modeled as a renewal process with increments distributed exponentially,

\begin{equation}
P(T_k-T_{k-1} > t | T_j, j\leq k-1) = \exp(-t \mu (T_{k-1}) )
\end{equation}

Here, 
\[
\mu (T_{k-1}) = \frac{1}{\dfrac{\beta(T_{k-1}) S(T_{k-1}) I(T_{k-1})}{N} + k E(T_{k-1}) + \gamma I(T_{k-1})}
\]

Based on the above stochastic model, \cite{chowell2004basic} provided Monte Carlo simulation, which shows good agreement with the actual data.

\subsection{Paper \cite{legrand2007understanding}}
Similar to \cite{chowell2004basic}, \cite{legrand2007understanding} also studies the Ebola outbreak in Congo 1995 and Uganda 2000. However, a major difference from \cite{chowell2004basic} is that, \cite{legrand2007understanding} modeled the spreading of disease in heterogeneous settings. In order to gain better insight of the epidemic dynamics, \cite{legrand2007understanding} subdivided the infectious phase into three stages:

\begin{itemize}
\item Transmission of infection in community setting (I)
\item Transmission of infection in hospital setting (H)
\item Transmission of infection after death during traditional burial (F)
\end{itemize}

The modified stochastic compartmental model is reproduced in Figure \ref{Fig:SEIHFR}. The transition rate among different stages are provided in the following stochastic model:

\begin{center}
\begin{tabular}{|c|c|c|}
\hline 
Transition $(i)$ & Effect & Transition Rate $(\lambda_i)$ \tabularnewline
\hline 
\hline 
1 & (S,E)$\to$(S-1,E+1) & $(\beta_{I}SI+\beta_{H}SH+\beta_{F}SF)/N$\tabularnewline
\hline 
2 & (E,I)$\to$(E-1,I+1) & $\alpha E$\tabularnewline
\hline 
3 & (I,H)$\to$(I-1,H+1) & $\gamma_{h}\theta_{1}I$\tabularnewline
\hline 
4 & (H,F)$\to$(H-1,F+1) & $\gamma_{dh}\delta_{2}H$\tabularnewline
\hline 
5 & (F,R)$\to$(F-1,R+1) & $\gamma_{f}F$\tabularnewline
\hline 
6 & (I,R)$\to$(I-1,R+1) & $\gamma_{i}(1-\theta)(1-\delta_{1})I$\tabularnewline
\hline 
7 & (I,F)$\to$(I-1,F+1) & $\delta_{1}(1-\theta_{1})\gamma_{d}I$\tabularnewline
\hline 
8 & (H,R)$\to$(H-1,R+1) & $\gamma_{ih}(1-\delta_{2})H$\tabularnewline
\hline 
\end{tabular}
\end{center}

Here,

\begin{eqnarray*}
\beta_I, \beta_H, \beta_F &=& \text{Transmission coefficient in community, hospital and funeral respectively}\\
\theta_1 &=& \text{fraction of infectious cases hospitalized}\\
\delta_1 &=& \text{fatality ratio of infectious}\\
\delta_2 &=& \text{fatality ratio of hospitalized patient}\\
&& \delta_1, \delta_2 \text{ are computed such that overall fatality ratio is $\delta$}\\
1/\alpha &=& \text{mean duration of incubation period}\\
1/\gamma_h &=& \text{mean duration from symptom onset to hospitalization}\\
1/\gamma_{i} &=& \text{mean duration of infectious period for survivors}\\
1/\gamma_{d} &=& \text{mean duration of infectious period to death}\\
1/\gamma_{dh} &=& \text{mean duration from hospitalization to death}\\
1/\gamma_{ih} &=& \text{mean duration from hospitalization to end of infectiousness for survivors}\\
1/\gamma_{f} &=& \text{mean duration from death to burial}\\
\end{eqnarray*}

In order to model the effect of interventions, a two step approach is used:

\begin{itemize}
\item Before intervention, population was exposed to the cases in community, hospitalization as well as funeral
\item After intervention, no transmission occurred at hospital or funeral, i.e. $\beta_H = \beta_F = 0$. The transmission coefficient in the community is decreased by a factor of $(1-z)$.
\end{itemize}

In the above mentioned model, parameters $(\beta_I, \beta_H, \beta_F , z)$ were estimated by fitting the model to the morbidity data of Congo 1995 and Uganda 2000 outbreak using approximate maximum likelihood. The estimates of other parameters in the above model were drawn from previous literatures.

Simulations of the stochastic model were performed using Gillespie's first reaction method \cite{gillespie1976general}. At each iteration of the algorithm, a time $\tau_i$ is drawn from an exponential distribution with parameter $\lambda_i$ for each of the transition. Here, $\lambda_i$ is the transition rate of the transition $i$. The next transition $\mu$ is the transition that has the minimum time to occurrence ($\tau_\mu$). Counts in each compartment are updated accordingly. In addition to the simulation result,  \cite{legrand2007understanding} also presented the basic reproductive rate as a function of $(\beta_I,\beta_H,\beta_F,\gamma_h,\gamma_{dh},\gamma_{ih},\gamma_d,\theta_1,\delta_1,\delta_2)$.


\subsection{Discussion on \cite{chowell2004basic, legrand2007understanding} and Network Model for Disease Spread \cite{newman2002spread}}
\cite{chowell2004basic, legrand2007understanding} as described in previous sections modeled the spread of Ebola using the compartmental modeling procedure. Even though \cite{legrand2007understanding} modified the original SIR model to reflect the heterogeneity of infection states, the underlying assumption is still the fully mixed model, where individual has an equal chance of spreading the disease to each other. However, disease like Ebola speeds through the populations via the networks formed by physical contacts among individuals. Models that incorporate network structure avoid the random-mixing assumption by assigning each individuals a finite set of permanent contacts to whom they can transmit the infection and from whom they can get infected. Although both in network and random-mixing based compartmental models, individuals may have the same number of contacts per unit time, within a network the set of contacts in fixed, whereas in random-mixing model, it is continually changing. A network model thus capture the permeance of interactions.



\bigskip 
{\em{Mike, Romit : please extend this section. I did not read \cite{newman2002spread} completely. we can add more stuff from that paper here, or any other paper that extend SIR models to networks}}

In \cite{newman2002spread}, the authors extend the concept of SIR model in network analysis.
They provide an exact solution to the SIR models of epidemic disease on networks of various kinds.
This is achieved using a combination of mapping to percolation models and generating function methods.

Transmissibility $T$ of a disease is defined as the average probability that an infectious
individual will transmit the disease to a susceptible individual with whom they have contact.
Epidemic threshold $T_c$ is the minimum transmissibility required for an outbreak to become
a large-scale epidemic. The authors provided the relation between the basic reproductive number
$R_0$ of an SIR network and the transmissibility $T$ as follows:


\[
some equation
\]

In addition, \cite{newman2002spread} also provided the value of epidemic the sold $T_c$. 

In an uncorrelated network, it is given by
\[
T_c =\dfrac{<k>}{<k^2> - <k>}
\]
Here, $<k>$ and $<k^2>$ are the mean degree and mean square degree of the network. Parameters for Poisson and power law networks are chosen such that all three networks share the same epidemic threshold.

The authors also predicted average size of the outbreak $<s>$ and probability of an epidemic $S$. 



\subsection{Paper \cite{meyers2005network}}
In our search for existing literatures in the area of network based model, however, we haven't come across any relevant work the spread of Ebola. In \cite{meyers2005network}, the authors model the spread of 2002-03 outbreak of SARS in Hong Kong and Canada using network model. A contact network model attempt to characterize every interpersonal contact that can potentially lead to disease transmission in community. Each person in community is represented as a node in the network and each contact between two people is represent as edge connecting the two nodes. The number of edges emanating from a node, that is the number of contacts a person has is called the degree of the node. The degree distribution of the node is a fundamental quantity in network theory.   In \cite{meyers2005network}, the authors presented three different contact network model for SARS outbreak:

\begin{itemize}
\item \textbf{Urban Network : }  A plausible contact network for an urban setting was generated using computer simulation based on the data for the city of Vancouver, British Columbia. $N=1000$ households were chosen at random from the Vancouver household size distribution statistics which yields approximately 2600 people. Household members were given ages based on age distribution statistics and are then assigned to schools according to school and class size distributions, assigned to occupation according to employment data, to hospitals as patients and caregivers according to the hospital employment and bed data and to other public places. 
\item \textbf{Random Network : } The above urban networks offers high degree of realism, however is quite complex. In addition to the urban model, a random network with Poisson degree distribution in which individuals connect to others independently and uniformly at random. 
\item \textbf{Scale-free Network: } There may be individuals in the network called ``superspreaders" with unusually large numbers of contacts or ``supershedders" who are unusually effective at excreting the virus into the environment they share with others. Neither the urban nor the random networks contains significant number of superspreaders. To incorporate the effects of superspreader in disease transmission, \cite{meyers2005network} also studies a network with truncated power law degree distribution. This type of network has a heavy tail of superspreaders. These individuals despite being few in numbers had profound effect on the outbreak patterns. 
\end{itemize}
 
 \cite{meyers2005network} extended the result from \cite{newman2002spread} to calculated the fate of an outbreak based on its initial condition, probability of a patient zero with degree $k$ will start an epidemic and the probability that outbreak of size $N$ will start an epidemic. However, \cite{newman2002spread} did not capture the temporal progression of the epidemic, rather provided overall number and distribution of infected individuals. The authors predicted the probability $S$ that an outbreak with $R_0>1$ will lead to an epidemic for there three networks. $S$ is often significantly less than one and can be different for two networks with the same $R_0$. Outbreaks are consistently less likely to reach and epidemic proportions in the power law networks than in the others. $R_0$ is a valuable epidemiological quantity. However, it has its limit since $R_0$ is a function of both the transmissibility of a disease and the contact patterns that underlie the transmission. Therefore, measuring $R_0$ in a location where contact rates are unusually high will lead to an estimate that is not appropriate for the larger community.  Estimating $T$ instead of $R_0$ give us a way out of this difficulty. 


\subsection{Paper \cite{keeling2005networks}}


\bigskip 
{\em{Mike, Romit : please extend this section if needed}}


\subsection{Paper \cite{meltzer2014estimating}}
\subsection{Paper \cite{myers2012information}}
\subsection{Chapter 21 from \cite{easley2010networks}}

\subsection{Paper \cite{newman2002spread}}

\bigskip
{\em{Mike, Romit : please extend this section if needed, I am not sure which of the paper we need to cite or describe}}

%%%%%%%%%%%%%%%%%%%%%%%%%%%%%%%%

\section{Discussion Project Proposal}
\label{sec:ProjectProposal}

\bigskip
{\em{Mike, Romit : please extend this section. I provided some outline and my ideas. We can add additional ideas. Please note that we may not need to do everything we listed here in the final project...but we can list our ideas here. Also some criticism of the existing papers, CDC report etc.}}


\begin{enumerate}
\item \textbf{Extension of \cite{chowell2004basic} and comparison with \cite{meltzer2014estimating}}: We can extend the framework described in \cite{chowell2004basic} for current Eboda data and estimate the parameters $\beta_0, \beta_1, k, q, \gamma$ as well as $R_0$. Based on the estimates, we can perform prediction of the epidemic. We can compare our prediction with the model proposed by CDC \cite{meltzer2014estimating} predicition.

\item \textbf{Extension of modified compartmental models relevant to Ebola for network analysis : } \cite{newman2002spread, meyers2005network} extended the SIR model to network analysis. Whether this model is readily applicable to the modified compartmental model used previously in Ebola research \cite{chowell2004basic, legrand2007understanding} is not completely clear to us. One possible research direction can be the extension of the modified compartmental model to network based analysis.

\item \textbf{Modeling the Ebola data for network analysis : } The spread of disease in Liberia is exponential whereas in other places like Guinea or Sierra Leon more linear. If we use compartmental model for analysis, then we will estimate different values of $R_0$. Another alternative approach could be to calculate a single value of transmissibility $T$ and then model the contacts networks for Liberia or Guinea using different different average degree of nodes for each individual. Alternatively, we can also try to use different types of network assumptions for different places.

\item \textbf{Can we model the contact network any better ?:} By taking into consideration demographic information of West Africa, Transportation Network, River flow (since the disease can also spread in case of traditional burial), Cell phone network data etc, we can cluster our contact network  for different district and then try to figure out the flow of disease information among different district.

\item \textbf{Predicting the future of the Ebola epidemic: } How to predict the future of the disease.... by using $R_0$, by using $T$, by using percolation theory etc...



\end{enumerate}

\subsection{Data Set}
\bigskip
{\em{Mike, Romit : please mention about the data set we are using....we may need to collect additional data since by the time we finish our project the existing data will be quite outdated. :)  please propose here how we are planning to collect those data....Does CDC website has any way ??. Please follow the guideline presented in the course websites...also reproduced below: }}


 The proposal should contain at least some amount of each of the following two types of content:

A test of a model or algorithm (that you have read about or your own) on a dataset or on simulated data
A proposal for a model or algorithm that potentially extends or improves the topics discussed in the papers you've read.
When writing the proposal you should try to answer the following questions:
What is the problem you are solving?
What data will you use (how will you get it)?
What work do you plan to do the project?
Which algorithms/techniques/models you plan to use/develop? Be as specific as you can!
Who will you evaluate your method? How will you test it? How will you measure success?
What do you expect to submit/accomplish by the end of the quarter?
Some other points to note:

The project should contain at least some amount of mathematical analysis, and some experimentation on real or synthetic data
The result of the project will typically be a 8 page paper, describing the approach, the results, and the related work.

%\begin{figure}[!ht]
%\centering
%\subfigure[]{
%\includegraphics[scale=0.42]{fig3a}
%\label{fig:subfig1_bpsk}}
%\subfigure[]{
%\includegraphics[scale=0.42]{fig3b}
%\l.abel{fig:subfig2_bpsk}}
%\subfigure[]{
%\includegraphics[scale=0.42]{fig3c}
%\label{fig:subfig3_bpsk}}
%\subfigure[]{
%\includegraphics[scale=0.42]{fig3d}
%\label{fig:subfig4_bpsk}}
%\caption[]{MMSE difference function $f_D(p,a)$ for different values of $a = e_i^2 \diagup b_i^2$ for (a) BPSK (b) QPSK (c) 16-QAM (d) 64-QAM.}
%\label{fig:mmseD_bpsk}
%\end{figure}
%


%%%%%%%%%%%%%%%%%%%%%%%%%%%%%%%%%%%%%%%%%%%
\appendices
\section{Proof of Proposition \ref{thm:secrecy}}


\bibliographystyle{IEEEtran}
\bibliography{bib_ref}


\end{document}
